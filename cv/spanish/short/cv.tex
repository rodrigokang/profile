\documentclass[10pt, letterpaper]{article}

% <><><><><><><><><><><><><><><><><><><><><><><><><><><><><><><><><><><>
% Paquetes
% <><><><><><><><><><><><><><><><><><><><><><><><><><><><><><><><><><><>

\usepackage[
    ignoreheadfoot,
    top=1.5cm,
    bottom=1.5cm,
    left=1.5cm,
    right=1.5cm,
    footskip=0.8cm
]{geometry}
\usepackage[explicit]{titlesec}
\usepackage{tabularx}
\usepackage{array}
\usepackage[dvipsnames]{xcolor}
\definecolor{primaryColor}{RGB}{0, 79, 144}
\usepackage{enumitem}
\usepackage{fontawesome5}
\usepackage{amsmath}
\usepackage[
    pdftitle={Rodrigo Kang},
    pdfauthor={Rodrigo Kang},
    pdfcreator={Rodrigo Kang},
    colorlinks=true,
    urlcolor=primaryColor
]{hyperref}
\usepackage[pscoord]{eso-pic}
\usepackage{calc}
\usepackage{bookmark}
\usepackage{lastpage}
\usepackage{changepage}
\usepackage{paracol}
\usepackage{ifthen}
\usepackage{needspace}
\usepackage{iftex}
\usepackage{graphicx}

\setlength{\parskip}{0.15cm}

% <><><><><><><><><><><><><><><><><><><><><><><><><><><><><><><><><><><>
% Fuentes y Codificación
% <><><><><><><><><><><><><><><><><><><><><><><><><><><><><><><><><><><>

\ifPDFTeX
    \input{glyphtounicode}
    \pdfgentounicode=1
    \usepackage[T1]{fontenc}
    \usepackage[utf8]{inputenc}
    \usepackage{lmodern}
\fi
\usepackage[default, type1]{sourcesanspro}

% <><><><><><><><><><><><><><><><><><><><><><><><><><><><><><><><><><><>
% Configuración de Página
% <><><><><><><><><><><><><><><><><><><><><><><><><><><><><><><><><><><>

\AtBeginEnvironment{adjustwidth}{\partopsep0pt}
\pagestyle{empty}
\setcounter{secnumdepth}{0}
\setlength{\parindent}{0pt}
\setlength{\topskip}{0pt}
\setlength{\columnsep}{0.5cm}

% <><><><><><><><><><><><><><><><><><><><><><><><><><><><><><><><><><><>
% Configuración del Pie de Página
% <><><><><><><><><><><><><><><><><><><><><><><><><><><><><><><><><><><>

\usepackage{datetime}
\newdateformat{monthyeardate}{\shortmonthname[\THEMONTH] \THEYEAR}

\makeatletter
\def\ps@customFooterStyle{%
    \def\@oddfoot{%
        \small\color{gray}%
        \raisebox{-0.1cm}[0pt][0pt]{%
            \makebox[\textwidth]{%
                \makebox[0pt][l]{\uppercase{\monthyeardate\today}}%
                \makebox[\textwidth]{\hfill\uppercase{RODRIGO KANG \, $\cdot$ \, CURRICULUM VITAE}\hfill}%
                \makebox[0pt][r]{\uppercase{\thepage}}%
            }%
        }%
    }%
    \def\@evenfoot{\@oddfoot}%
    \def\@oddhead{}%
    \def\@evenhead{}%
}
\makeatother

\pagestyle{customFooterStyle}

% <><><><><><><><><><><><><><><><><><><><><><><><><><><><><><><><><><><>
% Estilo de Sección
% <><><><><><><><><><><><><><><><><><><><><><><><><><><><><><><><><><><>

\titleformat{\section}{
    \needspace{3\baselineskip}
    \large\color{primaryColor}
}{
}{
}{
    \textbf{#1}
}[]

\titlespacing{\section}{-1pt}{0.2cm}{0.15cm}

% <><><><><><><><><><><><><><><><><><><><><><><><><><><><><><><><><><><>
% Entornos Personalizados
% <><><><><><><><><><><><><><><><><><><><><><><><><><><><><><><><><><><>

\newenvironment{highlights}{
    \begin{itemize}[
        topsep=0.08cm,
        parsep=0.08cm,
        partopsep=0pt,
        itemsep=0pt,
        leftmargin=0.4cm + 10pt
    ]
}{
    \end{itemize}
}

\newenvironment{onecolentry}{
    \begin{adjustwidth}{0cm}{0cm}
}{
    \end{adjustwidth}
}

% <><><><><><><><><><><><><><><><><><><><><><><><><><><><><><><><><><><>
% Configuración de Hipervínculos
% <><><><><><><><><><><><><><><><><><><><><><><><><><><><><><><><><><><>

\let\hrefWithoutArrow\href
\renewcommand{\href}[2]{\hrefWithoutArrow{#1}{\ifthenelse{\equal{#2}{}}{ }{#2 }\raisebox{.15ex}{\footnotesize \faExternalLink*}}}

% <><><><><><><><><><><><><><><><><><><><><><><><><><><><><><><><><><><>
% Documento
% <><><><><><><><><><><><><><><><><><><><><><><><><><><><><><><><><><><>

\begin{document}

% Encabezado con nombre, profesión, contacto y foto
\noindent
\begin{minipage}[c]{0.65\textwidth}
    \raggedright
    \fontsize{22pt}{22pt}\selectfont\textbf{Rodrigo Kang}

    {\normalsize \textbf{\textcolor{primaryColor}{Physicist \,|\, Data Scientist}}}
    
    \vspace{0.25cm}
    
    \footnotesize
    \mbox{{\faMapMarker*}\hspace*{0.13cm}Ciudad de Buenos Aires}%
    \kern 0.2cm | \kern 0.2cm%
    \mbox{\hrefWithoutArrow{mailto:rodrigokang88@gmail.com}{{\faEnvelope[regular]}\hspace*{0.13cm}rodrigokang88@gmail.com}}%
    
    \vspace{0.15cm}
    
    \mbox{\hrefWithoutArrow{http://www.linkedin.com/in/rodrigojuankang}{{\faLinkedinIn}\hspace*{0.13cm}rodrigojuankang}}%
    \kern 0.2cm | \kern 0.2cm%
    \mbox{\hrefWithoutArrow{https://github.com/rodrigokang/profile}{{\faGithub}\hspace*{0.13cm}rodrigokang}}%
\end{minipage}
\hfill
\begin{minipage}[c]{0.3\textwidth}
    \centering
    \includegraphics[width=2.8cm, keepaspectratio]{profile.jpg}
\end{minipage}

% Línea separadora
\vspace{0.2cm}
\noindent\color{primaryColor}\rule{\textwidth}{0.8pt}
\vspace{0.3cm}

% Contenido en dos columnas
\vspace{-0.8cm} 
\setcolumnwidth{0.68\textwidth, 0.3\textwidth}
\begin{paracol}{2}

% Columna principal (más ancha)
\begin{leftcolumn}

% Experiencia Laboral
\section{Experiencia Laboral}

\begin{onecolentry}
\textbf{\textcolor{black}{Líder de Ciencia de Datos}} \hfill {\footnotesize \textcolor{gray}{Mar 2025 - Presente}} \\ 
\textit{\textcolor{gray}{YPF Tecnología S.A. (Y-TEC)}}

\vspace{-0.05cm}

{\small \textcolor{black}{
Dirección del diseño y desarrollo de soluciones basadas en datos en la frontera de la I+D industrial, integrando aprendizaje automático, aprendizaje por refuerzo, optimización, simulación numérica y computación cuántica para abordar desafíos complejos en el sector O\&G.}
}
\end{onecolentry}

\vspace{0.05cm}

\begin{onecolentry}
\textbf{\textcolor{black}{Supervisor de CRM}} \hfill {\footnotesize \textcolor{gray}{Ene 2025 - Mar 2025}} \\ 
\textit{\textcolor{gray}{Cencosud S.A.}}

\vspace{0.05cm}

{\small \textcolor{black}{
Dirección del equipo de CRM de Científicos de Datos y Analistas de Datos para la mejora de la segmentación de clientes, predicción de abandono y sistemas de recomendación.}
}
\end{onecolentry}

\vspace{0.05cm}

\begin{onecolentry}
\textbf{\textcolor{black}{Data Scientist}} \hfill {\footnotesize \textcolor{gray}{Jul 2023 - Dic 2024}} \\ 
\textit{\textcolor{gray}{Cencosud S.A.}}

\vspace{0.05cm}

{\small \textcolor{black}{
Desarrollo de modelos predictivos mediante la combinación de consultas SQL complejas en AWS Redshift con algoritmos de aprendizaje automático en Python y AWS SageMaker.}
}
\end{onecolentry}

\vspace{0.05cm}

\begin{onecolentry}
\textbf{\textcolor{black}{Gas Trading \& Operations Research Analyst}} \hfill {\footnotesize \textcolor{gray}{Abr 2020 - Jul 2023}} \\ 
\textit{\textcolor{gray}{ECS S.A. - ENGIE GROUP}}

\vspace{0.05cm}

{\small \textcolor{black}{
Diseño e implementación de estrategias comerciales y algoritmos de optimización discreta utilizando técnicas de investigación operativa para el mercado de gas natural.}
}
\end{onecolentry}

\vspace{0.15cm}

% Formación Académica
\section{Formación Académica}

\begin{onecolentry}
\textbf{\textcolor{black}{Licenciatura en Física}} \hfill {\footnotesize \textcolor{gray}{Dic 2021}} \\ 
\textit{\textcolor{gray}{Universidad Nacional de La Plata}}

\vspace{0.05cm}

{\small \textcolor{black}{
Promedio de graduación de 8.40/10. Especialización en Teoría de Control Óptimo Cuántico.}
}
\end{onecolentry}

\vspace{0.05cm}

\begin{onecolentry}
\textbf{\textcolor{black}{Doctorado en Física (no finalizado)}} \hfill {\footnotesize \textcolor{gray}{Dic 2024}} \\ 
\textit{\textcolor{gray}{Universidad Nacional de La Plata}}

\vspace{0.05cm}

{\small \textcolor{black}{
Investigación avanzada en física hadrónica, enfocándose en los mecanismos de producción de piones mediante dispersión de neutrinos.}
}
\end{onecolentry}

\vspace{0.15cm}

% Proyectos
\section{Proyectos}

% Ciencia de Datos
\begin{onecolentry}
\textbf{\textcolor{black}{Ciencia de Datos}} \hfill {\footnotesize \textcolor{gray}{Presente}} \\
\textit{\textcolor{gray}{Proyectos de análisis de datos, modelado predictivo y ML, DL, RL e IA}}

\vspace{0.1cm}

{\small \textcolor{black}{
Los proyectos exploran temas como el análisis de clientes, el modelado financiero, la predicción de abandono, la ingeniería de características y la interpretabilidad de modelos, conectando la información basada en datos con la toma de decisiones estratégicas.}
}
\end{onecolentry}

\vspace{0.2cm}

% Computación Cuántica
\begin{onecolentry}
\textbf{\textcolor{black}{Computación Cuántica}} \hfill {\footnotesize \textcolor{gray}{Presente}} \\
\textit{\textcolor{gray}{Proyectos de algoritmos cuánticos, simulaciones y optimización cuántica}}

\vspace{0.1cm}

{\small \textcolor{black}{Proyectos centrados en aplicaciones de algoritmos cuánticos para resolver problemas complejos, simular sistemas cuánticos y proporcionar conocimientos en la intersección de la física, las matemáticas y la informática.}
}
\end{onecolentry}

\vspace{0.15cm}

\end{leftcolumn}

% Columna lateral (más estrecha)
\begin{rightcolumn}

% Resumen de Perfil (sin título)
\begin{onecolentry}
\raggedright
\small
\textcolor{black}{Físico y científico de datos con amplia experiencia en aprendizaje automático, aprendizaje por refuerzo, teoría de control óptimo cuántico y modelado computacional. Poseo una sólida formación en probabilidad, estadística, álgebra lineal, matemáticas avanzadas, computación cuántica y física computacional. Capaz de gestionar proyectos escalables y de apoyar la toma de decisiones basada en datos mediante soluciones innovadoras, consistentes y efectivas.}
\end{onecolentry}

\vspace{0.2cm}

% Habilidades - Lenguajes de Programación
\section{Lenguajes de Programación}

\begin{onecolentry}
\raggedright
\small
\faPython \; \textcolor{black}{\textbf{Python}} \\
\textit{\textcolor{gray}{NumPy, SciPy, Pandas, Scikit-learn, PyTorch, TensorFlow}}

\vspace{0.1cm}

\faDatabase \; \textcolor{black}{\textbf{SQL}} \\
\textit{\textcolor{gray}{PostgreSQL, AWS Redshift}}

\vspace{0.1cm}

\faAws \, \faMicrosoft \; \textcolor{black}{\textbf{AWS \& Azure}} \\
\textit{\textcolor{gray}{SageMaker, Redshift, Azure ML, Azure AI}}

\vspace{0.1cm}

\faLinux \; \textcolor{black}{\textbf{Linux}} \\
\textit{\textcolor{gray}{Bash scripting}}

\vspace{0.1cm}

\faGithub \; \textcolor{black}{\textbf{Git}} \\
\textit{\textcolor{gray}{Control de versiones}}

\vspace{0.1cm}

\textbf{\textcolor{black}{Otros}} \\
\textit{\textcolor{gray}{\LaTeX, R, C++}}
\end{onecolentry}

\vspace{0.2cm}

% Idiomas
\section{Idiomas}

\begin{onecolentry}
\raggedright
\small
\textbf{\textcolor{black}{Español}} \\
\textit{\textcolor{gray}{Competencia nativa}}

\vspace{0.1cm}

\textbf{\textcolor{black}{Inglés}} \\
\textit{\textcolor{gray}{Competencia profesional (C1)}}
\end{onecolentry}

\vspace{0.2cm}

% Licencias y Certificaciones
\section{Licencias y Certificaciones}

\begin{onecolentry}
\raggedright
\small
\textbf{\textcolor{black}{Azure AI Fundamentals}} \\
\textit{\textcolor{gray}{Microsoft, Feb 2025}}

\vspace{0.1cm}

\textbf{\textcolor{black}{Masterclass en IA}} \\
\textit{\textcolor{gray}{Udemy, Feb 2025}}

\vspace{0.1cm}

\textbf{\textcolor{black}{Finanzas Cuantitativas}} \\
\textit{\textcolor{gray}{UCEMAx, Nov 2023}}

\vspace{0.1cm}

\textbf{\textcolor{black}{Probabilidad y Estadística}} \\
\textit{\textcolor{gray}{UNLP, Dic 2021}}
\end{onecolentry}

\end{rightcolumn}

\end{paracol}

\end{document}